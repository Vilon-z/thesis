\cabstract{
   我国正在进行射电望远镜的建设工作,对于射电望远镜的建设,选址至关重要。本文使用可搬移式电磁信号监测设备在广东省清远市连山县及西沙群岛两处潜在射电站点进行射电环境监测,评估选址处的射频干扰( Radio Frequency Interference ,RFI),进而分析这两处作为射电望远镜台址用于探测21厘米氢线的可行性。通过采集的射电环境监测数据进行频谱分析,可以发现这两处环境信号干扰较弱,低频射电环境较为干净,适宜作为射电望远镜台址,并且可以在台址附近建立射电宁静区,来减少无线电通信对于射电望远镜工作的影响。
}
\ckeywords{射电天文,电磁环境,频谱监测,数据处理,天文选址}

\eabstract{
   China is currently conducting the construction of radio telescopes, for which site selection plays a important role. This study employs portable electromagnetic signal monitoring equipment to conduct radio environment assessments at two potential radio telescope sites—Lianshan County in Qingyuan City, Guangdong Province, and the Xisha Islands—to evaluate radio frequency interference (RFI) and analyze the feasibility of these locations as suitable sites for detecting the 21 cm hydrogen line. Through spectrum analysis of the collected radio environment monitoring data, it was found that both sites exhibit low levels of signal interference and a relatively clean low-frequency radio environment, rendering them appropriate for hosting radio telescopes. Furthermore, establishing radio quiet zones near these sites is proposed to minimize the impact of wireless communications on telescope operations.
}
\ekeywords{Radio astronomy, Electromagnetic environment, Spectrum monitoring, Data processing, Astronomical site selection}

